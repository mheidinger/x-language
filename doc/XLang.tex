\documentclass[a4paper]{article}
\usepackage{inputenc}[utf8]
\usepackage{fullpage} % Package to use full page
\usepackage{parskip} % Package to tweak paragraph skipping
\usepackage{tikz} % Package for drawing
\usepackage{amsmath}
\usepackage{hyperref}
\usepackage{amssymb}
\usepackage{csquotes}
\usepackage{listings}

\lstset{
    language=Matlab,
    basicstyle=\ttfamily\small,
    numbers=left,
    stepnumber=1,
    showstringspaces=false,
    tabsize=2,
    breaklines=true,
    breakatwhitespace=false,
		columns=fixed,
}

\title{XLang}
\author{Max Heidinger, Pascal Riesinger}
\date{\today}

\begin{document}

\maketitle


\section{Einleitung}

XLang ist eine Programmiersprache entwickelt im Rahmen der Vorlesung \textit{Compilerbau} an der
Dualen Hochschule Baden-Württemberg Karlsruhe.

\section{Syntax}

\subsection{Grammatik}

Die folgende Grammatik in der Extended-Backus-Naur-Form beschreibt die Syntax von XLang.
Wiederholungen, die in geschweiften Klammern (\texttt{\{ \}}) gekennzeichnet sind, schließen null Wiederholungen (das leere Wort) mit ein.

\lstinputlisting{../ebnf.txt}


\subsection{Beschreibung}

\subsubsection{Struktur}

Ein XLang Programm ist in 3 Teile aufgeteilt.
Diese werden durch eine Folge von Gleichheitszeichen (\texttt{=}) voneinander abgetrennt, welche
mindestens 3 Zeichen lang sein muss.
Zunächst wird der Name des Programmes vermerkt.
Anschließend folgt der Variablendeklarationsteil, in welchem alle im Programmteil verwendeten
Variablen deklariert werden müssen.
Der dritte Teil des Programmes ist der sogenannte Programmteil, welcher alle Instruktionen
beinhaltet.

Leerzeichen können überall im Programm eingefügt werden und werden primär verwendet, um
Schlüsselwörter von anderen Zeichen zu trennen.

\subsubsection{Variablen}

Es gibt in XLang drei Datentypen:

\begin{itemize}
	\item Ganzzahlen, deklariert durch den Typ \texttt{int}.
	\item Gleitkommazahlen, deklariert durch den Typ \texttt{float}.
	\item Zeichenfolgen, deklariert durch den Typ \texttt{string}. Diese dürfen maximal 1024 Zeichen
		lang sein.
\end{itemize}

Wie bereits angemerkt müssen alle im Programm verwendeten Variablen im Variablenteil delariert
werden.
Da XLang ohne Funktionen und Unterprogramme auskommt, müssen Variablen, welche als Eingabeparameter
fungieren sollen, mit einer sogenannten \texttt{ImportFlag} gekennzeichnet werden. Eine Deklaration
einer Eingabevariable sieht dann bespielsweise wie folgt aus: \texttt{-> name string}.

Die Eingabevariablen werden beim Programmstart aus der Kommandozeile ausgelesen.
Dabei ist die Reihenfolge der Deklaration gleich der Reihenfolge der Übergabe.

Variablen, deren Wert am Ende des Programmes ausgegeben werden soll, müssen mit einer
\texttt{ExportFlag} gekennzeichnet werden. Eine Deklaration einer Ausgabevariable sieht dann
beispielsweise wie folgt aus: \texttt{<- ergebnis float}.
Die Ausgabevariablen werden in der Reihenfolge ausgegeben, in welcher sie deklariert wurden.

Es gibt keine Möglichkeit, Variablen während der Ausführung auszugeben oder einzulesen.

Variablen können Literale, also konstante Werte oder Ausdrücke zugewiesen werden.
Ein Ausdruck ist bei numerischen Werten entweder eine andere Variable oder eine mathematische
Formel. Bei Rechnungen können die Grundrechenarten (Addition, Subtraktion, Multiplikation und
Division). verwendet werden. Hierbei gilt die gleiche Priorisierung, wie im C-Standard.

Bei Zeichenketten ist ein Ausdruck entweder eine andere Variable oder eine Verkettung von mehreren
Literalen und Variablen über den Additionsoperator.

\subsubsection{Konditionen}

Die Konditionalausdrücke in XLang folgen dem Beispiel anderer Programmiersprachen.
Wenn der Vergleich nach dem \texttt{if} zu einem logisch \enquote{wahren} Wert evaluiert, werden
alle Anweisungen im folgenden Anweisungsblock ausgeführt.
Der \texttt{if}-Anweisung können optional beliebig viele \texttt{else if}-Anweisungen folgen.
Diese werden nur ausgewertet, wenn der Vergleich der voranstehenden Anweisung zu einem unwahren Wert
evaluiert wurde.
Abgeschlossen werden kann ein Konditionalausdruck durch ein \texttt{else}. Die Anweisungen im
\texttt{else}-Block werden ausgeführt, wenn kein vorheriger Vergleich im Konditionalausdruck zu
einem wahren Wert evaluiert werden konnte.

\subsubsection{Kommentare}

Es gibt in XLang zwei Kommentarformen, welche an jeder Stelle im Quelltext eingefügt werden können.
Zeilenkommentare erstrecken sich vom Kommentaranfang (\texttt{//}) bis zum nächsten Zeilenende.
Mehrzeilige Kommentare erstrecken sich vom Kommentaranfang (\texttt{/*}) bis zum nächsten
Komenntarende (\texttt{*/}).
Beide Kommentartypen können jegliche Zeichen beinhalten.
Kommentare werden vom Compiler ignoriert und haben keine Bedeutung für das Programm.

\subsubsection{Schleifen}

Es gibt in XLang zwei Arten von Schleifen. Eine \texttt{while}-Schleife führt dein Codeblock so
lange aus, bis die Bedingung im Kopf der Schleife zu einem unwahren Wert evaluiert wird.
While-Schleifen sind kopfgesteuert.

Die zweite-Art der Schleifen in XLang sind \texttt{for}-Schleifen. Diese erlauben es, eine Zuweisung
im Fuß der Schleife durchzuführen. Zunächst wird eine Variable zugewiesen, welche dann in der
Kondition in der Schleife verwendet wird. Der dritte Teil der Schleife weißt der Variable einen
neuen Wert zu. Vor jedem Durchlauf wird der Vergleich durchgeführt, dann der Programmblock, falls
der Vergleich zu einem wahren Wert evaluiert, nach dem Ausführen des Programmblockes wird dann die
Zuweisung ausgeführt.


\section{Verwendetes Allgemeinwissen}

In XLang wird verschiedenes Allgemeinwissen verwendet.
Ein Beispiel ist, dass eine deklarierte Variable im Code verwendet werden kann.
Auch das Verhalten der Kontrollstrukturen folgt den \enquote{Regeln} des Allgemeinwissen.
Bei numerischen Berechnungen werden die allgemein bekannten Rechenregeln angewandt.


\section{Beispielprogramme}

\subsection{Berechnung der Fakultät}

\lstinputlisting{../factorial.x}

\subsection{Berechnung des größten gemeinsamen Teilers}

\lstinputlisting{../ggt.x}

\section{BNF}

\lstinputlisting{../bnf.txt}

\end{document}
