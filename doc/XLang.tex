\documentclass[a4paper]{article}
\usepackage{inputenc}[utf8]
\usepackage{fullpage} % Package to use full page
\usepackage{parskip} % Package to tweak paragraph skipping
\usepackage{tikz} % Package for drawing
\usepackage{amsmath}
\usepackage{hyperref}
\usepackage{amssymb}
\usepackage{csquotes}
\usepackage{listings}

\lstset{
	tabsize=2
}

\title{XLang}
\author{Max Heidinger, Pascal Riesinger}
\date{\today}

\begin{document}

\maketitle


\section{Einleitung}

XLang ist eine Programmiersprache entwickelt im Rahmen der Vorlesung \textit{Compilerbau} an der
Dualen Hochschule Baden-Württemberg Karlsruhe.

\section{Syntax}

\subsection{Grammatik}

Die folgende Grammatik in der Backus-Naur-Form beschreibt die Syntax von XLang.
Die Regeln folgen diesen Konventionen:

\begin{itemize}
	\item Der Aufbau ist \texttt{Symbol ::= Ersetzungen.}, jede Regel wird also mit einem Punkt
		beendet.
	\item Terminalsymbole werden zwischen Hochkommata gesetzt, alle anderen Symbole sind
		Nichtterminale.
	\item Ausnahmen zu obiger Regel sind das \enquote{Pipe}-Symbol, runde Klammern und der Punkt. Wie vorher
		beschrieben beendet der Punkt eine Regel. Das Pipe-Symbol bezeichnet alternative Ersetzungen.
		Runde Klammern werden zur Gruppierung verwendet, um die Lesbarkeit zu verbessern.
	\item Leerzeichen zwischen Nichtterminalen, sowie Terminalen und Nichtterminalen sind zu
		ignorieren. Falls Leerzeichen explizit notwending sind, ist dies durch ein Terminal der Form
		\texttt{' '} gekennzeichnet.
\end{itemize}

\lstinputlisting{../bnf.txt}


\subsection{Beschreibung}

\subsubsection{Struktur}

Ein XLang Programm ist in 3 Teile aufgeteilt.
Diese werden durch eine Folge von Gleichheitszeichen \texttt{=} voneinander abgetrennt, welche
mindestens 3 Zeichen lang sein muss.
Zunächst wird der Name des Programmes vermerkt.
Anschließend folgt der Variablendeklarationsteil, in welchem alle im Programmteil verwendeten
Variablen deklariert werden müssen.
Der dritte Teil des Programmes ist der sogenannte Programmteil, welcher alle Instruktionen
beinhaltet.

\subsubsection{Variablen}

Es gibt in XLang drei Datentypen:

\begin{itemize}
	\item Ganzzahlen, deklariert durch den Typ \texttt{int}.
	\item Gleitkommazahlen, deklariert durch den Typ \texttt{float}.
	\item Zeichenfolgen, deklariert durch den Typ \texttt{string}.
\end{itemize}

Wie bereits angemerkt müssen alle im Programm verwendeten Variablen im Variablenteil delariert
werden.
Da XLang ohne Funktionen und Unterprogramme auskommt, müssen Variablen, welche als Eingabeparameter
fungieren sollen mit einem sogenannten \texttt{ImportFlag} gekennzeichnet werden. Eine Deklaration
einer Eingabevariable sieht dann bespielsweise wie folgt aus: \texttt{-> name string}.

Die Eingabevariablen werden beim Programmstart aus der Kommandozeile ausgelesen.
Dabei ist die Reihenfolge der Deklaration gleich der Reihenfolge der Übergabe.

Variablen, deren Wert am Ende des Programmes ausgegeben werden soll, müssen mit einem
\texttt{ExportFlag} gekennzeichnet werden. Eine Deklaration einer Ausgabevariable sieht dann
beispielsweise wie folgt aus: \texttt{<- ergebnis float}.
Die Ausgabevariablen werden in der Reihenfolge ausgegeben, in welcher sie deklariert wurden.

Es gibt keine Möglichkeit, Variablen während der Ausführung auszugeben oder einzulesen.







\end{document}
